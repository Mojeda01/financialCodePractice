\documentclass{article}

\usepackage{amsmath}
\usepackage{graphicx}

\title{Partial Derivatives}

\begin{document}
\maketitle
\clearpage

In the realms of calculus, a derivative is a fundamental concept that represents the rate at which a function
changes at any given point. It is the mathematical embodiment of the notion of instantaneous velocity,
capturing how a quantity varies with respect to change in another quantity. Formally, if we consider a 
real-valued $f$ defined on an interval in the real line, the derivative of $f$ at a point $a$ within this
interval, denoted as $f'(a)$, is defined as the limit:

$$
f'(a)=\text{lim}_{h\rightarrow 0}\frac{f(a+h)-f(a)}{h}
$$

Providing that this limit exist. This definition encapslates the idea of the slope of the tangent line to
the graph of the function $f$ at the point $a$, thus providing a geometric interpretation of the 
derivative.

When the function under scrutiny depends on multiple variables, the concept of the derivative generalizes
to partial derivatives. A partial derivative of a function of several variables is its derivative with
respect to one of those variables, with the others held constant. If $f$ is a function of $x$ and $y$, then 
the partial derivative of $f$ with respect to $x$ at the point $(a,b)$, denoted as 
$\frac{\partial f}{\partial x}(a,b)$, is defined as:

$$
\frac{\partial f}{\partial x}(a, b) = \text{lim}_{h \leftarrow 0} \frac{f(a+h,b)-f(a,b)}{h}
$$

Analogously, the partial derivative with respect to $y$ is defined by:

$$
\frac{\partial f}{\partial y}(a, b) = \text{lim}_{h->0}\frac{f(a, b+h)}{h}
$$

These definitions reflect the rate of change of the function $f$ in the direction of each coordinate
axis. In the same spirit as the derivative in one dimension, partial derivatives in multiple
dimensions can be interpreted as the slopes of the tangent plane to the surface defined by the function
$f(x,y)$ at the point $(a,b)$

Consider the function $f(x,y)=x^2y+3xy^2$. The partial derivative of $f$ with respect to $x$ is computed
by differentiating $f$ with respect to $x$ while treating $y$ as a constant:

$$
\frac{\partial f}{\partial x} = 2xy + 3y^2
$$

Similarly, differentiating $f$ with respect to $y$ while holding $x$ constant yields:

$$
\frac{\partial f}{\partial y} = x^2 + 6xy
$$

These expressions quantify the gradient of $f$ in the directions of the $x$-axis and $y$-axis,
respectively.

In mathematical analysis, the concepts of the derivative and partial derivative serve as cornerstones
for differential calculus, enabling the examination of functions and their behaviors in both 
one-dimensional and multidimensional spaces. These constructs underpin numerous applications across 
diverse scientific disciplines, including physics, engineering, and economics, where they facilitate
the modeling and analysis of dynamic systems and the exploration of rates of change and slopes of 
functions.

\end{document}
